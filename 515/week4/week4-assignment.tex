% Options for packages loaded elsewhere
\PassOptionsToPackage{unicode}{hyperref}
\PassOptionsToPackage{hyphens}{url}
%
\documentclass[
]{article}
\usepackage{amsmath,amssymb}
\usepackage{iftex}
\ifPDFTeX
  \usepackage[T1]{fontenc}
  \usepackage[utf8]{inputenc}
  \usepackage{textcomp} % provide euro and other symbols
\else % if luatex or xetex
  \usepackage{unicode-math} % this also loads fontspec
  \defaultfontfeatures{Scale=MatchLowercase}
  \defaultfontfeatures[\rmfamily]{Ligatures=TeX,Scale=1}
\fi
\usepackage{lmodern}
\ifPDFTeX\else
  % xetex/luatex font selection
\fi
% Use upquote if available, for straight quotes in verbatim environments
\IfFileExists{upquote.sty}{\usepackage{upquote}}{}
\IfFileExists{microtype.sty}{% use microtype if available
  \usepackage[]{microtype}
  \UseMicrotypeSet[protrusion]{basicmath} % disable protrusion for tt fonts
}{}
\makeatletter
\@ifundefined{KOMAClassName}{% if non-KOMA class
  \IfFileExists{parskip.sty}{%
    \usepackage{parskip}
  }{% else
    \setlength{\parindent}{0pt}
    \setlength{\parskip}{6pt plus 2pt minus 1pt}}
}{% if KOMA class
  \KOMAoptions{parskip=half}}
\makeatother
\usepackage{xcolor}
\usepackage[margin=1in]{geometry}
\usepackage{color}
\usepackage{fancyvrb}
\newcommand{\VerbBar}{|}
\newcommand{\VERB}{\Verb[commandchars=\\\{\}]}
\DefineVerbatimEnvironment{Highlighting}{Verbatim}{commandchars=\\\{\}}
% Add ',fontsize=\small' for more characters per line
\usepackage{framed}
\definecolor{shadecolor}{RGB}{248,248,248}
\newenvironment{Shaded}{\begin{snugshade}}{\end{snugshade}}
\newcommand{\AlertTok}[1]{\textcolor[rgb]{0.94,0.16,0.16}{#1}}
\newcommand{\AnnotationTok}[1]{\textcolor[rgb]{0.56,0.35,0.01}{\textbf{\textit{#1}}}}
\newcommand{\AttributeTok}[1]{\textcolor[rgb]{0.13,0.29,0.53}{#1}}
\newcommand{\BaseNTok}[1]{\textcolor[rgb]{0.00,0.00,0.81}{#1}}
\newcommand{\BuiltInTok}[1]{#1}
\newcommand{\CharTok}[1]{\textcolor[rgb]{0.31,0.60,0.02}{#1}}
\newcommand{\CommentTok}[1]{\textcolor[rgb]{0.56,0.35,0.01}{\textit{#1}}}
\newcommand{\CommentVarTok}[1]{\textcolor[rgb]{0.56,0.35,0.01}{\textbf{\textit{#1}}}}
\newcommand{\ConstantTok}[1]{\textcolor[rgb]{0.56,0.35,0.01}{#1}}
\newcommand{\ControlFlowTok}[1]{\textcolor[rgb]{0.13,0.29,0.53}{\textbf{#1}}}
\newcommand{\DataTypeTok}[1]{\textcolor[rgb]{0.13,0.29,0.53}{#1}}
\newcommand{\DecValTok}[1]{\textcolor[rgb]{0.00,0.00,0.81}{#1}}
\newcommand{\DocumentationTok}[1]{\textcolor[rgb]{0.56,0.35,0.01}{\textbf{\textit{#1}}}}
\newcommand{\ErrorTok}[1]{\textcolor[rgb]{0.64,0.00,0.00}{\textbf{#1}}}
\newcommand{\ExtensionTok}[1]{#1}
\newcommand{\FloatTok}[1]{\textcolor[rgb]{0.00,0.00,0.81}{#1}}
\newcommand{\FunctionTok}[1]{\textcolor[rgb]{0.13,0.29,0.53}{\textbf{#1}}}
\newcommand{\ImportTok}[1]{#1}
\newcommand{\InformationTok}[1]{\textcolor[rgb]{0.56,0.35,0.01}{\textbf{\textit{#1}}}}
\newcommand{\KeywordTok}[1]{\textcolor[rgb]{0.13,0.29,0.53}{\textbf{#1}}}
\newcommand{\NormalTok}[1]{#1}
\newcommand{\OperatorTok}[1]{\textcolor[rgb]{0.81,0.36,0.00}{\textbf{#1}}}
\newcommand{\OtherTok}[1]{\textcolor[rgb]{0.56,0.35,0.01}{#1}}
\newcommand{\PreprocessorTok}[1]{\textcolor[rgb]{0.56,0.35,0.01}{\textit{#1}}}
\newcommand{\RegionMarkerTok}[1]{#1}
\newcommand{\SpecialCharTok}[1]{\textcolor[rgb]{0.81,0.36,0.00}{\textbf{#1}}}
\newcommand{\SpecialStringTok}[1]{\textcolor[rgb]{0.31,0.60,0.02}{#1}}
\newcommand{\StringTok}[1]{\textcolor[rgb]{0.31,0.60,0.02}{#1}}
\newcommand{\VariableTok}[1]{\textcolor[rgb]{0.00,0.00,0.00}{#1}}
\newcommand{\VerbatimStringTok}[1]{\textcolor[rgb]{0.31,0.60,0.02}{#1}}
\newcommand{\WarningTok}[1]{\textcolor[rgb]{0.56,0.35,0.01}{\textbf{\textit{#1}}}}
\usepackage{longtable,booktabs,array}
\usepackage{calc} % for calculating minipage widths
% Correct order of tables after \paragraph or \subparagraph
\usepackage{etoolbox}
\makeatletter
\patchcmd\longtable{\par}{\if@noskipsec\mbox{}\fi\par}{}{}
\makeatother
% Allow footnotes in longtable head/foot
\IfFileExists{footnotehyper.sty}{\usepackage{footnotehyper}}{\usepackage{footnote}}
\makesavenoteenv{longtable}
\usepackage{graphicx}
\makeatletter
\def\maxwidth{\ifdim\Gin@nat@width>\linewidth\linewidth\else\Gin@nat@width\fi}
\def\maxheight{\ifdim\Gin@nat@height>\textheight\textheight\else\Gin@nat@height\fi}
\makeatother
% Scale images if necessary, so that they will not overflow the page
% margins by default, and it is still possible to overwrite the defaults
% using explicit options in \includegraphics[width, height, ...]{}
\setkeys{Gin}{width=\maxwidth,height=\maxheight,keepaspectratio}
% Set default figure placement to htbp
\makeatletter
\def\fps@figure{htbp}
\makeatother
\setlength{\emergencystretch}{3em} % prevent overfull lines
\providecommand{\tightlist}{%
  \setlength{\itemsep}{0pt}\setlength{\parskip}{0pt}}
\setcounter{secnumdepth}{-\maxdimen} % remove section numbering
\ifLuaTeX
  \usepackage{selnolig}  % disable illegal ligatures
\fi
\IfFileExists{bookmark.sty}{\usepackage{bookmark}}{\usepackage{hyperref}}
\IfFileExists{xurl.sty}{\usepackage{xurl}}{} % add URL line breaks if available
\urlstyle{same}
\hypersetup{
  pdftitle={week4\_assignment},
  pdfauthor={Kohei Nishitani},
  hidelinks,
  pdfcreator={LaTeX via pandoc}}

\title{week4\_assignment}
\author{Kohei Nishitani}
\date{2024-02-06}

\begin{document}
\maketitle

{
\setcounter{tocdepth}{2}
\tableofcontents
}
\hypertarget{section-1-description-of-the-data}{%
\subsection{Section 1: Description of the
data}\label{section-1-description-of-the-data}}

This
\href{https://www.kaggle.com/datasets/harlfoxem/housesalesprediction/data}{dataset
originates from King County, WA, USA}, contains a wide range of real
estate sales data which is comprehensive details on homes sold within
the county, featuring over 20 columns that captures various aspects of
the propertes listed. The primary objective of utilizing this dataset is
to analyze the real estate market trends in King County, focusing on
factors that influence property prices. By examining attributes such as
the size of the living space, number of bedrooms and bathrooms, and
additional features like waterfront views and grade of the house, we can
identify patterns and insights that are crucial for buyers, sellers, and
investors in the housing market.

\hypertarget{section-2-reading-the-data-into-r.}{%
\subsection{Section 2: Reading the data into
R.}\label{section-2-reading-the-data-into-r.}}

\begin{Shaded}
\begin{Highlighting}[]
\CommentTok{\# set working directory}
\FunctionTok{setwd}\NormalTok{(}\StringTok{"D:/Workspace/McDaniel{-}Repository/515/week4"}\NormalTok{)}

\CommentTok{\# read csv and assign this to data variable, and specify the header}
\NormalTok{data }\OtherTok{\textless{}{-}} \FunctionTok{read.csv}\NormalTok{(}\StringTok{\textquotesingle{}kc\_house\_data.csv\textquotesingle{}}\NormalTok{,}\AttributeTok{header=}\ConstantTok{TRUE}\NormalTok{)}
\end{Highlighting}
\end{Shaded}

\hypertarget{section-3-clean-the-data}{%
\subsection{Section 3: Clean the data}\label{section-3-clean-the-data}}

\begin{Shaded}
\begin{Highlighting}[]
\NormalTok{cln\_data }\OtherTok{\textless{}{-}}\NormalTok{ data }\SpecialCharTok{\%\textgreater{}\%} 
  \FunctionTok{mutate}\NormalTok{(}
  \CommentTok{\# these variables should be factor not number data type}
  \AttributeTok{id =} \FunctionTok{as.factor}\NormalTok{(id),}
  \AttributeTok{waterfront =} \FunctionTok{as.factor}\NormalTok{(waterfront),}
  \AttributeTok{condition =} \FunctionTok{as.factor}\NormalTok{(condition), }
  \AttributeTok{view =} \FunctionTok{as.factor}\NormalTok{(view),}
  \AttributeTok{zipcode =} \FunctionTok{as.factor}\NormalTok{(zipcode),}
  \AttributeTok{yr\_renovated =} \FunctionTok{if\_else}\NormalTok{(yr\_renovated }\SpecialCharTok{==} \DecValTok{0}\NormalTok{, }\ConstantTok{NA\_integer\_}\NormalTok{, yr\_renovated),}
  \AttributeTok{bedrooms =} \FunctionTok{as.numeric}\NormalTok{(bedrooms),}
  \AttributeTok{sqft\_living =} \FunctionTok{as.numeric}\NormalTok{(sqft\_living),}
  \AttributeTok{sqft\_living15 =} \FunctionTok{as.numeric}\NormalTok{(sqft\_living15),}
  \AttributeTok{sqft\_lot =} \FunctionTok{as.numeric}\NormalTok{(sqft\_lot),}
  \AttributeTok{sqft\_lot15 =} \FunctionTok{as.numeric}\NormalTok{(sqft\_lot15),}
  \AttributeTok{sqft\_above =} \FunctionTok{as.numeric}\NormalTok{(sqft\_above),}
  \AttributeTok{sqft\_basement =} \FunctionTok{as.numeric}\NormalTok{(sqft\_basement),}
  \AttributeTok{yr\_built =} \FunctionTok{as.numeric}\NormalTok{(yr\_built),}

\NormalTok{  ) }\SpecialCharTok{\%\textgreater{}\%}
  \CommentTok{\# at least less than 100 years old houses}
  \FunctionTok{filter}\NormalTok{(yr\_built }\SpecialCharTok{\textgreater{}} \DecValTok{1924}\NormalTok{) }\SpecialCharTok{\%\textgreater{}\%}
  \CommentTok{\# select few relevant columns for convenience}
  \FunctionTok{select}\NormalTok{(price, bedrooms, bathrooms, waterfront, grade)}
\end{Highlighting}
\end{Shaded}

\hypertarget{section-4-characteristics-of-the-data}{%
\subsection{Section 4: Characteristics of the
data}\label{section-4-characteristics-of-the-data}}

This real estate sales data contains rows and variables like (id, date,
price, bedrooms, bathrooms, sqft\_living, sqft\_lot, floors, waterfront,
view, condition, grade, sqft\_above, sqft\_basement, yr\_built,
yr\_renovated, zipcode, lat, long, sqft\_living15, sqft\_lot15).\\
This dataframe has 21613 rows and 21 columns. The names of the columns
and a brief description of each are in the table below:

\begin{longtable}[]{@{}
  >{\raggedleft\arraybackslash}p{(\columnwidth - 4\tabcolsep) * \real{0.0631}}
  >{\raggedright\arraybackslash}p{(\columnwidth - 4\tabcolsep) * \real{0.0541}}
  >{\raggedright\arraybackslash}p{(\columnwidth - 4\tabcolsep) * \real{0.8829}}@{}}
\caption{Column ID, Names and Descriptions}\tabularnewline
\toprule\noalign{}
\begin{minipage}[b]{\linewidth}\raggedleft
Column Number
\end{minipage} & \begin{minipage}[b]{\linewidth}\raggedright
Column Name
\end{minipage} & \begin{minipage}[b]{\linewidth}\raggedright
Description
\end{minipage} \\
\midrule\noalign{}
\endfirsthead
\toprule\noalign{}
\begin{minipage}[b]{\linewidth}\raggedleft
Column Number
\end{minipage} & \begin{minipage}[b]{\linewidth}\raggedright
Column Name
\end{minipage} & \begin{minipage}[b]{\linewidth}\raggedright
Description
\end{minipage} \\
\midrule\noalign{}
\endhead
\bottomrule\noalign{}
\endlastfoot
1 & price & Price of each home sold \\
2 & bedrooms & \# of bedrooms \\
3 & bathrooms & \# of bathrooms \\
4 & waterfront & A dummy variable for whether the apartment was
overlooking the waterfront or not \\
5 & grade & An index from 1 to 13, where 1-3 falls short of building
construction and design, 7 has an average level of construction and
design, and 11-13 have a high quality level of construction and
design \\
\end{longtable}

\hypertarget{section-5-subset-and-summary}{%
\subsection{Section 5 Subset and
Summary}\label{section-5-subset-and-summary}}

\begin{verbatim}
##   Avg_Price Min_Price Max_Price Avg_Bedrooms Min_Bedrooms Max_Bedrooms
## 1  534435.3     75000   7062500     3.393391            0           33
##   Avg_Grade Min_Grade Max_Grade
## 1  7.720437         1        13
\end{verbatim}

\end{document}
